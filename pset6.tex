\documentclass[fleqn,12pt]{article}
\input{preamble2}
\usepackage{amsmath}
\usepackage[colorlinks]{hyperref}
\setlength{\parindent}{0em}
\setlength{\parskip}{1em}
\begin{document}
\thispagestyle{empty}

\section*{Logic pset 6}

Resources: HLW \href{https://www.jstor.org/stable/j.ctvxrpz0q.6}{Ch 3}
and
\href{https://princeton.hosted.panopto.com/Panopto/Pages/Viewer.aspx?id=3be71c4d-ebf5-4d02-9101-accf01077417}{Lecture 8}

Rubric: Each proof is graded out of 4 points, according to the following scheme:
\begin{description}
  \item[4 points] absolutely perfect
\item[3 points] the proof may contain a few misapplications of rules that could easily be remedied by, for example, adding a step. The most common mistake here is skipping a step of double negation elimination.
\item[2 points] Some good thinking went into the argument, but as it
  stands, it couldn’t easily be converted into a valid proof. This
  category is mostly applicable to long proofs, where there are good
  sub-arguments, but where the overall strategy is misguided.
\item[1 point] something coherent was written.
\item[0 points] nothing coherent was written. \end{description}

Problems: You may use any of the rules of inference, including reductio ad
absurdum, to prove the following sequents.
\begin{enumerate}
\item $P\to Q\:\vdash\: \neg (P\wedge\neg Q)$
\item $(P\wedge Q)\to \neg Q\:\vdash\: P\to \neg Q$
\item $P\to\neg P\:\vdash\:\neg P$
\item $\neg (P\to Q)\:\vdash\: P\wedge\neg Q$
\end{enumerate}     

\end{document}


%%% Local Variables:
%%% mode: latex
%%% TeX-master: t
%%% End:
