\documentclass[fleqn,12pt]{article}
\input{preamble2}
\usepackage{amsmath}
\usepackage[colorlinks]{hyperref}
\setlength{\parindent}{0em}
\setlength{\parskip}{1em}
\begin{document}
\thispagestyle{empty}

\section*{Logic pset 3}

Resources: HLW \href{https://www.jstor.org/stable/j.ctvxrpz0q.5}{Ch
  2}, \href{https://princeton.hosted.panopto.com/Panopto/Pages/Viewer.aspx?id=bab527ef-2604-4956-af35-acbd0180a259}{Lecture 4} and \href{https://princeton.hosted.panopto.com/Panopto/Pages/Viewer.aspx?id=85dd66c8-78cd-4799-b131-acbe01605214}{Lecture 5}

\begin{enumerate}
\item Prove that the following argument forms are valid.  The premises
  are to the left of the $\vdash$ symbol, the conclusion is to the
  right.  You should number the lines of your proof, and each line
  must either be a premise (i.e.\ an assumption) or be justified by
  one of the following rules of inference: $\wedge$I, $\wedge$E,
  $\vee$I, MP, MT, or DN.
\begin{enumerate}
\item $P\to (Q\to R),\,P\to Q,\,P\:\vdash\: R$
\item $P\:\vdash\: (P\vee R)\wedge (P\vee Q)$
\item $P\:\vdash\: Q\vee (\neg\neg P\vee R)$  
\item $\neg\neg Q\to P,\,\neg P\:\vdash\:\neg Q$
\item $Q\to (P\to R),\,\neg R\wedge Q\:\vdash\: \neg P$
\end{enumerate} \bigskip 

\item Explain what is wrong with the following ``proof''.
\[ \begin{array}{c l l l}
     (1) & P\vee (Q\wedge R) & & \text{A}   \\
     (2) & P\vee Q & & 1 \wedge\text{E} \end{array} \] 
\end{enumerate}     

\end{document}


%%% Local Variables:
%%% mode: latex
%%% TeX-master: t
%%% End:
