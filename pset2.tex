\documentclass[fleqn,12pt]{article}
\usepackage[colorlinks]{hyperref}
\input{preamble2}
\setlength{\parindent}{0em}
\setlength{\parskip}{1em}
\begin{document}
\thispagestyle{empty}

\section*{Logic pset 2}

Resources: \href{https://www.jstor.org/stable/j.ctvxrpz0q.5}{Ch 2} of HLW and \href{https://princeton.hosted.panopto.com/Panopto/Pages/Viewer.aspx?id=ca5a58cd-b0e1-47a7-9fb1-acbd00ba6a46}{Lecture
  3}

Instructions: Represent the propositional structure of each of the
following sentences.  First identify the atomic component sentences
(i.e.\ sentences that do not contain connectives) and abbreviate each
with a distinct capital letter.  We have suggested letters after the
sentences.  Then represent the form of the original sentence using the
symbols $\vee ,\wedge ,\neg ,\to$ for the connectives ``or'', ``and'',
``not'', ``if\dots then\dots ''. Make sure to include parentheses, if
necessary to disambiguate.

\begin{enumerate}
\item It's not true that if Ron doesn't do his homework then Hermione
  will finish it for him.  (R,H)
\item Harry will be singed unless he evades the dragon's fiery breath.
  (S,E)
\item Aristotle was neither a great philosopher nor a great scientist.
  (P,S)
\item Mark will get an A in logic only if he does the homework or
bribes the professor. (A,H,B)
\item Dumbledore will be killed, and either McGonagall will become
  head of school and Hogwarts will flourish, or else it won't
  flourish. (D,M,H)
\item Harry and Dumbledore are not both right about the moral status
  of Professor Snape.  (H,D)
\end{enumerate}


\end{document}

%%% Local Variables:
%%% mode: latex
%%% TeX-master: t
%%% End:
